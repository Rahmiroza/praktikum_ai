\documentclass[12pt,a4paper]{article}
\usepackage[left=3.00cm, right=2.00cm, bottom=2.00cm, top=3.00cm]{geometry}
\begin{document}
\title{Artificial Intelegence}
\maketitle


\section{Lusia Violita Aprilian/1164080}
\subsection{Artificial Intelegence}
\begin{enumerate}
	\item Pengertian AI\\
	Menurut Minsky, Kecerdasan Buatan ialah suatu ilmu yang mempelajari cara membuat 			komputer melakukan sesuatu seperti yang dilakukan oleh manusia. Lalu menurut Ensiklopedi Britannica, Kecerdasan Buatan ialah cabang ilmu komputer yang merepresentasi pengetahuan lebih banyak menggunakan symbol-simbol daripada bilangan, dan memproses informasi berdasarkan metode heuristic atau berdasarkan jumlah aturan.

	Menurut Stuart J. Russell & Peter Norvig, Kecerdasan Buatan ialah perangkat komputer yang dapat memahami lingkungannya dan dapat mengambil tindakan yang memaksimalkan peluang kesuksesan di lingkungan tersebut untuk beberapa tujuan.

Berdasarkan beberapa teori tentang kecerdasan buatan diatas, penulis menyimpulkan bahwa kecerdasan buatan adalah suatu ilmu yang membuat sebuah mesin menjadi cerdas, sehingga kecerdasan mesin tersebut mirip dengan kecerdasan manusia serta dapat mengambil keputusan sendiri untuk menyelesaikan sebuah masalah.

	\item Sejarah dan Perkembangan
	\begin{itemize}
		\item 1941 ( Era Komputer Elektronik )\\
    Pada era ini, telah ditemukan pertama kali yakni alat penyimpanan dan pemrosesan informasi atau disebut komputer elektronik. Ini juga digunakan untuk dasar pengembangan program ke arah AI.
    \item 1943 – 1956 ( Era Persiapan AI )\\
    Pada tahun 1943, terdapat dua peneliti yakni Warren McCulloch dan Walter Pitts yang berhasil membuat sebuah model tiruan dari tiap neuron seperti on dan off. Mereka membuktikan bahwa setiap fungsi dapat dihitung dengan suatu jaringan sel saraf dan semua hubungan logis bisa diimplementasikan dengan struktur jaringan yang sederhana.\\
    Pada tahun 1950, Norbert Wiener melakukan penelitian tentang prinsip teori feedback. Bentuk implementasi dari penelitian tersebut salah satunya adalah thermostat.\\
    Pada tahun 1956, John McCarthy mencoba meyakinkan Minsky, Claude Shannon, dan Nathaniel Rochester untuk membantunya dalam melakukan penelitian di bidang automata, jaringan saraf, dan pembelajaran intelijensia. Mereka mengerjakan proyek ini kurang lebih selama 2 bulan di Universitas Dartmouth. Hasilnya adalah berupa program yang mampu berpikir non-numerik dan menyelesaikan masalah pemikiran, yang disebut Principia Mathematica. Berdasarkan hal ini, telah ditentukan bahwa McCarthy disebut sebagai father of Artificial Intelligence/ Bapak Kecerdasan Buatan.
    \item 1952 – 1969 ( Awal Perkembangan )\\
    Pada tahun 1958, McCarthy di MIT AI Lab mengeluarkan bahasa pemrograman tingkat tinggi yaitu LISP, dimana sekarang sudah mulai sering digunakan dalam pembuatan program-program AI. Lalu, McCarthy membuat program yang disebut programs with common sense. Di program tersebut, dibuat sebuah rancangan untuk menggunakan pengetahuan dalam mencari solusi dari sebuah masalah.\\
    Pada tahun 1959, Program komputer bernama General Problem Solver berhasil dibuat oleh Herbert A. Simon, J.C. Shaw, dan Allen Newell. Program tersebut dirancang untuk memulai proses penyelesaian masalah secara manusiawi. Pada tahun yg sama Nathaniel Rochester dari IBM dan para mahasiswanya merilis sebuah program AI yaitu geometry theorem prover. Program ini dapat mebuktikan bahwa suatu teorema menggunakan axioma-axioma yang ada.\\
    Pada tahun 1963, program yang dibuat oleh James Slagle bisa menyelesaikan masalah integral tertutup untuk mata kuliah Kalkulus.\\
    Pada tahun 1968, program analogi buatan Tom Evan dapat menyelesaikan masalah analogi geometri yang ada pada tes IQ.
    \item 1966 – 1974 ( Perkembangan AI Lambat )\\
    Perkembangan AI mulai melambat pada tahun 1966 – 1974, yang disebabkan adanya beberapa kesulitan yang di hadapi seperti  Program-program AI yang bermunculan hanya mengandung sedikit atau bahkan tidak mengandung sama sekali pengetahuan pada subjeknya, banyak terjadi kegagalan pada pembuatan program AI, serta terdapat beberapa batasan pada struktur dasar yang digunakan untuk menghasilkan perilaku intelijensia.
    \item 1969 – 1979 (Sistem berbasis pengetahuan )\\
    Pada tahun 1960, Ed Feigenbaum, Bruce Buchanan, dan Joshua Lederberg mulai merintis proyek bernama DENDRAL yaitu program yang digunakan untuk memecahkan masalah struktur molekul dari informasi yang didapatkan dari spectometer massa. Dari segi diagnosa medis juga terdapat  yang menemukan sistem berbasis Ilmu pengetahuan, yaitu Saul Amarel dalam proyek computer ini biomedicine. Proyek ini diawali dari keinginan untuk mendapatkan diagnosa penyakit berdasarkan pengetahuan yang ada pada mekanisme penyebab proses penyakit.
    \item 1980 – 1988 ( AI menjadi Industri )\\
    Industralisasi AI diawali dengan ditemukannya sebuah sistem pakar yang dinamakan R1 yang mampu mengkonfigurasi sistem-sistem komputer baru. Program tersebut mulai dijalankan di Digital Equipment Corporation (DEC), McDermott, pada tahun 1982. Pada tahun 1986, program ini telah berhasil menghemat biaya sebesar US 40 juta per tahun.\\
    Pada tahun 1988, kelompok AI di DEC menjalankan program sistem pakar sebanyak 40 sistem pakar. Hampir semua perusahaan besar di USA mempunyai divisi Ai sendiri yang menggunakan maupun mempelajari sistem pakar itu sendiri. Industri AI yang sedang ramai diperbincangkan juga melibatkan perusahaan-perusahaan besar seperti Carnegie Group, Inference, IntelliCorp, dan Technoledge yang menawarkan software tools untuk membangun sistem pakar. Perusahaan bidang hardware seperti LISP Machines Inc., Texas Instruments, Symbolics, dan Xerox dan lain-lain juga ikut berperan dalam membangun sebuah workstation yang dioptimasi untuk pembangunan program LISP. Sehingga, perusahaan yang sudah berdiri sejak tahun 1982 hanya menghasilkan beberapa juta US dollar per tahun meningkat menjadi 2 milyar US dollar per tahun pada tahun 1988.
    \item 1986 – sekarang ( kembalinya jaringan saraf tiruan )
    Meskipun bidang ilmu komputer, telah melakukan penolakan terhadap jaringan saraf tiruan setelah diterbitkannya sebuah buku berjudul ‘Perceptrons’ karangan Minsky dan Papert, tetapi para ilmuwan masih terus mempelajari bidang ilmu tersebut dari sudut pandang yang lain, yaitu bidang fisika. Ahli fisika seperti Hopfield (1982) menggunakan teknik-teknik mekanika statistika untuk menganalisa sifat-sifat penyimpanan dan optimasi pada jaringan saraf. ahli psikolog, David Rumhelhart dan Geoff Hinton melanjutkan penelitian tersebut tentang model jaringan saraf pada memori. Pada tahun 1985-an sedikitnya empat kelompok riset menemukan algoritma Back-Propagation. Algoritma inipun akhirnya berhasil diimplementasikan ke dalam ilmu bidang komputer dan psikologi.
	\end{itemize}		
\end{enumerate}


\subsection{Supervised Learning dan Data}
\begin{enumerate}
	\item Supervised Learning\\
	Sebuah pendekatan dengan kondisi dimana sudah ada terdapat kumpulan data yang dilatih atau ditraining, dan terdapat beberapa variabel yang sudah ditentukan sehingga tujuan dari pendekatan tersebut mengarah ke data yang sudah ada.
	
	\item Klasifikasi\\
	Sebuah sejenis program yang dapat menentukan objek yang ada termasuk jenis apa berdasarkan variabel-variabel yang sudah ditentukan.
	
	\item Regresi\\
	Sebuah metode yang digunakan untuk menentukan dan memprediksi berdasarkan hubungan sebab – akibat antara satu variabel ke variabel lainnya.
	
	\item Unsupervised Learning\\
	Sebuah pendekatan yang dimana tidak memiliki data yang dilatih, namun ingin di kelompokkan berdasarkan beberapa variabel dengan kemauan sendiri.
	
	\item Data Set\\
	Objek yang merepresentasikan data dan relasi didalam memori.

	\item Training Set\\
	Himpunan dari berbagai pasangan objek, kelas yang dapat menunjukkan objek tersebut yang sudah diberi label.

	\item Testing Set\\
	Himpunan data yang sudah berlabel lain, yang digunakan untuk mengukur persentase sampel yang diklasifikan dengan benar atau persentase sampel mengalami kesalahan.

\end{enumerate}
\end{document}