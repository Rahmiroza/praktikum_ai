\documentclass[12pt,a4paper]{article}
\usepackage[left=3.00cm, right=2.00cm, bottom=2.00cm, top=3.00cm]{geometry}
\begin{document}
\title{Resume Tentang Data}
\maketitle

\section{Pengertian}
\subsection{Supervised Learning}
Sebuah pendekatan dengan kondisi dimana sudah ada terdapat kumpulan data yang dilatih atau ditraining, dan terdapat beberapa variabel yang sudah ditentukan sehingga tujuan dari pendekatan tersebut mengarah ke data yang sudah ada.

\subsection{Klasifikasi}
Sebuah sejenis program yang dapat menentukan objek yang ada termasuk jenis apa berdasarkan variabel-variabel yang sudah ditentukan.

\subsection{Regresi}
Sebuah metode yang digunakan untuk menentukan dan memprediksi berdasarkan hubungan sebab – akibat antara satu variabel ke variabel lainnya.

\subsection{Unsupervised Learning}
Sebuah pendekatan yang dimana tidak memiliki data yang dilatih, namun ingin di kelompokkan berdasarkan beberapa variabel dengan kemauan sendiri.

\subsection{Data Set}
Objek yang merepresentasikan data dan relasi didalam memori.

\subsection{Training Set}
Himpunan dari berbagai pasangan objek, kelas yang dapat menunjukkan objek tersebut yang sudah diberi label.

\subsection{Testing Set}
Himpunan data yang sudah berlabel lain, yang digunakan untuk mengukur persentase sampel yang diklasifikan dengan benar atau persentase sampel mengalami kesalahan.

\end{document}